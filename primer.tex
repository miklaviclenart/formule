\documentclass{formule}

\usepackage{tabularray}

\begin{document}

\subsection*{Stožnice}
\(A x^2 + B xy + C y^2 + D x + E y + F = 0\) \\ \(D = B^2 - 4 A C\) \\
\(D < 0\): elipsa (če \(A = C\) in \(B = 0\), krožnica) \\
\(D = 0\): parabola \\
\(D > 0\): hiperbola

\subsection*{Limite}
\(\lim_{t \to 0} t^{\alpha} \cdot \log t^{2k} = 0\) (za \(\alpha > 0, k \in \mathbb{Z}\)) \\
\(\lim_{t \to 0^{+}} t^{\alpha} \cdot \log t^{\beta} = 0\) (za \(\alpha > 0\)) \\
\(\lim_{t \to 0} t^{\alpha} \cdot \sin t^{- \beta} = 0\) (za \(\alpha, \beta \in \mathbb{Z}^{+}\)) \\
\(\lim_{t \to 0} t^{- \alpha} \cdot \sin t^{\beta} =
\begin{cases}
    0 & \alpha < \beta \\
    1 & \alpha = \beta \\
    \pm \infty & \alpha > \beta
\end{cases}\)

\subsection*{Kotne funkcije}
\vspace*{-1.1em}% multicols for some reason da ful prostora na vrh
\begin{multicols}{2}
\(\sin (\frac{\pi}{2} - x) = \cos x\) \\
\(\cos (\frac{\pi}{2} - x) = \sin x\) \\
\(\tan (\frac{\pi}{2} - x) = \cot x\) \\
\(\cot (\frac{\pi}{2} - x) = \tan x\)
\end{multicols}
\vspace*{-1.1em}

\subsection*{Adicijski izreki}
\(\sin \alpha + \sin \beta = 2 \sin \frac{\alpha + \beta}{2} \cos \frac{\alpha - \beta}{2}\) \\
\(\cos \alpha + \cos \beta = 2 \cos \frac{\alpha + \beta}{2} \cos \frac{\alpha - \beta}{2}\) \\
\(\cos \alpha - \cos \beta = - 2 \sin \frac{\alpha + \beta}{2} \sin \frac{\alpha - \beta}{2}\) \\
\(\sin \alpha \cos \beta = \frac{1}{2} (\sin (\alpha + \beta) + \sin (\alpha - \beta))\) \\
\(\cos \alpha \cos \beta = \frac{1}{2} (\cos (\alpha + \beta) + \cos (\alpha - \beta))\) \\
\(\sin \alpha \sin \beta = \frac{1}{2} (\cos (\alpha - \beta) - \cos (\alpha + \beta))\) \\
\(\sin (x \pm y) = \sin x \cos y \pm \cos x \sin y\) \\
\(\cos (x \pm y) = \cos x \cos y \mp \sin x \sin y\) \\
\(\tan (x \pm y) = (\tan x \pm \tan y) / (1 \mp \tan x \tan y)\) \\
\(\sin 2x = 2 \sin x \cos x = 2 \tan x / (1 + \tan^2 x)\) \\
\(\cos 2x = \cos^2 x - \sin^2 x = 2 \cos^2 x - 1 \\ = 1 - 2 \sin^2 x = (1 - \tan^2 x) / (1 + \tan^2 x)\) \\
\(\tan 2x = 2 \tan x / (1 - \tan^2 x)\)

\subsection*{Vrednosti}
{\small
\begin{tblr}{
  colspec = {l *{6}{c}},
  % hlines,
  %row{1} = {font=\bfseries},
  stretch = 1.2 % a bit more vertical spacing
}
           & \(0\)                 & \(\pi/12\)                      & \(\pi/6\)           & \(\pi/4\)           & \(\pi/3\)           & \(5\pi/12\)            & \(\pi/2\)        \\
\(\sin\)  & \(0\)                 & \(\frac{\sqrt{6} - \sqrt{2}}{4}\) & \(\frac{1}{2}\)     & \(\frac{\sqrt{2}}{2}\) & \(\frac{\sqrt{3}}{2}\) & \(\frac{\sqrt{6} + \sqrt{2}}{4}\) & \(1\)            \\
\(\cos\)  & \(1\)                 & \(\frac{\sqrt{6} + \sqrt{2}}{4}\) & \(\frac{\sqrt{3}}{2}\) & \(\frac{\sqrt{2}}{2}\) & \(\frac{1}{2}\)      & \(\frac{\sqrt{6} - \sqrt{2}}{4}\) & \(0\)            \\
\(\tan\)  & \(0\)                 & \(2 - \sqrt{3}\)                & \(\frac{\sqrt{3}}{3}\) & \(1\)                 & \(\sqrt{3}\)         & \(2 + \sqrt{3}\)          & \(\pm \infty\)   \\
\end{tblr}
}

\subsection*{Inverzne kotne funkcije}
\(\arcsin : [-1, 1] \to [- \pi / 2, \pi / 2] \\ \text{ODV: } 1 / \sqrt{1 - x^2}, \text{INT: } x \arcsin x + \sqrt{1 - x^2}\) \\
\(\arccos : [-1, 1] \to [0, \pi] \\ \text{ODV: } -1 / \sqrt{1 - x^2}, \text{INT: } x \arccos x - \sqrt{1 - x^2}\) \\
\(\arctan : (- \infty, \infty) \to (- \pi / 2, \pi / 2) \\ \text{ODV: } 1 / (x^2 + 1), \text{INT: } x \arctan x - 1/2 \cdot \ln (x^2 + 1)\)

\subsection*{Integrali}
\(\int u dv = uv - \int v du\) \\
\(\int a^x dx = a^x / \ln a\) \\
\(\int e^{kx} dx = e^{kx} / k\) \\
\(\int \sin^2 x dx = - 1 / 4 (\sin (2x) - 2x), [0, 2 \pi] \to \pi\) \\
\(\int \cos^2 x dx = 1 / 2 (\cos x \sin x + x), [0, 2 \pi] \to \pi \)\\
\(\int dx / \sin^2 x = - \cot x\) \\
\(\int dx / \cos^2 x = - \tan x\) \\
\(\int f'(x) / f(x) dx = \ln | f(x) |\) \\
\(\int x e^2 dx = e^x (x- 1)\) \\
\(\int \log_a x dx = x \log_a x - x / \ln a\) \\
\(\int dx / (x^2 + a^2) = 1/a \cdot \arctan (x/a)\) \\
\(\int dx / (x^2 - a^2) = 1/2a \cdot \ln | (x - a) / (x + a) |\) \\
\(\int dx / (a^2 - x^2) = 1/2a \cdot \ln | (a + x) / (a - x) |\) \\
\(\int dx / \sqrt{x^2 \pm a} = \ln | x + \sqrt{x^2 \pm a} |\) \\
\( \int dx / \sqrt{a - x^2} = \arctan (x / \sqrt{a - x^2}) = \arcsin (x / \sqrt{a}) \) \\
\(\int x \cdot dx / \sqrt{x^2 - a} = \sqrt{x^2 - a} \) \\
\(\int dx / \sqrt{a x^2 + bx + c} = \\
\begin{cases}
    1 / \sqrt{a} \cdot \ln | 2ax + b + 2 \sqrt{a} \sqrt{ax^2 + bx + c} | & \text{; } a > 0 \\
    - 1 / \sqrt{- a} \arcsin ((2ax + b) / \sqrt{D}) & \text{; } a < 0
\end{cases}\)
\(\int e^{ax} \sin (bx) dx = e^{ax} / (a^2 + b^2) \cdot (a \sin (bx) - b \cos (bx))\) \\
\(\int e^{ax} \cos (bx) dx = e^{ax} / (a^2 + b^2) \cdot (a \cos (bx) + b \sin (bx))\)

\subsection*{Substitucije}
\(u = \tan (x / 2), \sin x = 2u / (1 + u^2), \cos x = (1-u^2) / (1 + u^2), dx = 2 du / (1 + u^2)\) \\
\(t = \tan x, \sin^2 x = t^2 / (1 + t^2), \cos^2 x = 1 / (1 + t^2), dx = dt / (1 + t^2)\)

\subsection*{Funkcije, preslikave}
\begin{definition}
    Naj bo \(D \subset \RRn\). Funkcija \(f \colon D \to \RR\) je v notranji točki \(a \in D\)
    \emph{diferenciabilna}, če obstaja tak vektor \(A \in \mathbb{R}^n\),
    da je \(f(a + h) = f(a) + A \cdot h + o(h)\), kjer je \(\lim_{h \to 0} o(h) / \| h \|| = 0\).
\end{definition}
\begin{theorem}
    Če je funkcija diferenciabilna v točki \(a \in D\), potem je tam zvezna in parcialno odvedljiva na vse spremenljivke. Tedaj velja
    \[A = \left | \frac{\partial f}{\partial x_1} (a), \frac{\partial f}{\partial x_2} (a), \ldots , \frac{\partial f}{\partial x_n} (a) \right |.\]
\end{theorem}
\begin{theorem}
    Če je funkcija zvezno parcialno odvedljiva na vse spremenljivke, je diferenciabilna.
\end{theorem}
\begin{theorem}
    Naj bo \(g \colon \mathbb{R}^m \to \mathbb{R}^n\) diferenciabilna v \(a\) in
    \(h \colon \mathbb{R}^n \to \mathbb{R}^p\) diferenciabilna v \(g (a)\). Potem je
    \(h \circ g\) diferenciabilna v \(a\) in velja \(D (h \circ g) (a) = D h (g (a)) \circ D g (a)\).
\end{theorem}
\begin{theorem}[Vpeljava novih spremenljivk]
    Naj bo \(f (x) = f(x_1, \ldots, x_n)\) diferenciabilna funkcija in \(g (y) = g(y_1, \ldots, y_n) = f (x (y))\).
    Potem velja
    \[\frac{\partial}{\partial x_k} = \sum_{i = 1}^{n} \frac{\partial y_i}{\partial x_k} \cdot \frac{\partial}{\partial y_i}.\]
    (\(x_k\) so spremenljivke prvotne funkcije, \(y_i (\ldots, x_k, \ldots)\) so funkcije (spremenljivke), ki jih vpeljemo)
\end{theorem}
\begin{definition}[Laplace]
    Za dvakrat zvezno odvedljivo funkcijo \(u (x, y)\) definiramo \(\triangle u = u_{xx} + u_{yy}\).
    V polarnih koordinatah: \(\triangle u = u_{rr} + 1 / r^2 u_{\varphi \varphi} + 1 / r u_{r}\).
    Dvakrat zvezno odvedljiva funkcija \(u\) je \emph{harmonična}, če velja \(\triangle u = 0\).
\end{definition}

\end{document}
